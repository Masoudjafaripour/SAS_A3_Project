\tikzstyle{jp}=[circle,fill=blue!40, inner sep=0pt, minimum size=4mm]
\matrix[square matrix](m){
      &   &   \\
      &   &   \\
      &   &   \\
};

\newcommand{\neigb} {
  \node[jp] at (m-2-2) {$\mathbf{c}$};
  \draw[->, thick, shorten <=5pt] (m-2-2.center) -- (m-2-1.center);
  \draw[->, thick, shorten <=5pt] (m-2-2.center) -- (m-2-3.center);
  \draw[->, thick, shorten <=5pt] (m-2-2.center) -- (m-1-2.center);
  \draw[->, thick, shorten <=5pt] (m-2-2.center) -- (m-3-2.center);

  \draw[->, thick, shorten <=5pt] (m-2-2.center) -- (m-1-1.center);
  \draw[->, thick, shorten <=5pt] (m-2-2.center) -- (m-1-3.center);
  \draw[->, thick, shorten <=5pt] (m-2-2.center) -- (m-3-1.center);
  \draw[->, thick, shorten <=5pt] (m-2-2.center) -- (m-3-3.center);
}

\newcommand{\cornercut} {
  \node[jp] at (m-2-2) {$\mathbf{c}$};
  \draw[->, thick, shorten <=5pt] (m-2-2.center) -- (m-2-1.center);
  \draw[->, thick, shorten <=5pt] (m-2-2.center) -- (m-2-3.center);
  \draw[->, thick, shorten <=5pt] (m-2-2.center) -- (m-3-2.center);

  \draw[->, thick, shorten <=5pt] (m-2-2.center) -- (m-3-1.center);
  \draw[->, thick, shorten <=5pt] (m-2-2.center) -- (m-3-3.center);
  \node[fill=black, inner sep=0pt, fit=(m-1-2)(m-1-2)] {};
}
